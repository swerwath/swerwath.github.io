\documentclass[twoside]{article}
\setlength{\oddsidemargin}{0.25 in}
\setlength{\evensidemargin}{-0.25 in}
\setlength{\topmargin}{-0.6 in}
\setlength{\textwidth}{6.5 in}
\setlength{\textheight}{8.5 in}
\setlength{\headsep}{0.75 in}
\setlength{\parindent}{0 in}
\setlength{\parskip}{0.1 in}

%
% ADD PACKAGES here:
%

\usepackage{amsmath,amsfonts,graphicx}

%
% The following commands set up the lecnum (lecture number)
% counter and make various numbering schemes work relative
% to the lecture number.
%
\newcounter{lecnum}
\renewcommand{\thepage}{\thelecnum-\arabic{page}}
\renewcommand{\thesection}{\thelecnum.\arabic{section}}
\renewcommand{\theequation}{\thelecnum.\arabic{equation}}
\renewcommand{\thefigure}{\thelecnum.\arabic{figure}}
\renewcommand{\thetable}{\thelecnum.\arabic{table}}

%
% The following macro is used to generate the header.
%
\newcommand{\lecture}[4]{
   \pagestyle{myheadings}
   \thispagestyle{plain}
   \newpage
   \setcounter{lecnum}{#1}
   \setcounter{page}{1}
   \noindent
   \begin{center}
   \framebox{
      \vbox{\vspace{2mm}
    \hbox to 6.28in { {\bf EE16B
	\hfill Spring 2016} }
       \vspace{4mm}
       \hbox to 6.28in { {\Large \hfill Controls  \hfill} }
       \vspace{2mm}
      \vspace{2mm}}
   }
   \end{center}
}
%
% Convention for citations is authors' initials followed by the year.
% For example, to cite a paper by Leighton and Maggs you would type
% \cite{LM89}, and to cite a paper by Strassen you would type \cite{S69}.
% (To avoid bibliography problems, for now we redefine the \cite command.)
% Also commands that create a suitable format for the reference list.
\renewcommand{\cite}[1]{[#1]}
\def\beginrefs{\begin{list}%
        {[\arabic{equation}]}{\usecounter{equation}
         \setlength{\leftmargin}{2.0truecm}\setlength{\labelsep}{0.4truecm}%
         \setlength{\labelwidth}{1.6truecm}}}
\def\endrefs{\end{list}}
\def\bibentry#1{\item[\hbox{[#1]}]}

%Use this command for a figure; it puts a figure in wherever you want it.
%usage: \fig{NUMBER}{SPACE-IN-INCHES}{CAPTION}
\newcommand{\fig}[3]{
			\vspace{#2}
			\begin{center}
			Figure \thelecnum.#1:~#3
			\end{center}
	}
% Use these for theorems, lemmas, proofs, etc.
\newtheorem{theorem}{Theorem}[lecnum]
\newtheorem{lemma}[theorem]{Lemma}
\newtheorem{proposition}[theorem]{Proposition}
\newtheorem{claim}[theorem]{Claim}
\newtheorem{corollary}[theorem]{Corollary}
\newtheorem{definition}[theorem]{Definition}
\newenvironment{proof}{{\bf Proof:}}{\hfill\rule{2mm}{2mm}}

% **** IF YOU WANT TO DEFINE ADDITIONAL MACROS FOR YOURSELF, PUT THEM HERE:

\newcommand\E{\mathbb{E}}

\begin{document}
%FILL IN THE RIGHT INFO.
%\lecture{**LECTURE-NUMBER**}{**DATE**}{**LECTURER**}{**SCRIBE**}
%\footnotetext{These notes are partially based on those of Nigel Mansell.}

% **** YOUR NOTES GO HERE:

% Some general latex examples and examples making use of the
% macros follow.  
%**** IN GENERAL, BE BRIEF. LONG SCRIBE NOTES, NO MATTER HOW WELL WRITTEN,
%**** ARE NEVER READ BY ANYBODY.
This lecture's notes illustrate some uses of
various \LaTeX\ macros.  
Take a look at this and imitate.

\section{Some theorems and stuff} % Don't be this informal in your notes!

We now delve right into the proof.

\begin{lemma}
This is the first lemma of the lecture.
\end{lemma}

\begin{proof}
The proof is by induction on $\ldots$.
For fun, we throw in a figure.
%%%NOTE USAGE !
\fig{1}{1in}{A Fun Figure}

This is the end of the proof, which is marked with a little box.
\end{proof}

\subsection{A few items of note}

Here is an itemized list:
\begin{itemize}
\item this is the first item;
\item this is the second item.
\end{itemize}

Here is an enumerated list:
\begin{enumerate}
\item this is the first item;
\item this is the second item.
\end{enumerate}

Here is an exercise:

{\bf Exercise:}  Show that ${\rm P}\ne{\rm NP}$.

Here is how to define things in the proper mathematical style.
Let $f_k$ be the $AND-OR$ function, defined by

\[ f_k(x_1, x_2, \ldots, x_{2^k}) = \left\{ \begin{array}{ll}

	x_1 & \mbox{if $k = 0$;} \\

	AND(f_{k-1}(x_1, \ldots, x_{2^{k-1}}),
	   f_{k-1}(x_{2^{k-1} + 1}, \ldots, x_{2^k}))
	 & \mbox{if $k$ is even;} \\

	OR(f_{k-1}(x_1, \ldots, x_{2^{k-1}}),
	   f_{k-1}(x_{2^{k-1} + 1}, \ldots, x_{2^k}))	
	& \mbox{otherwise.} 
	\end{array}
	\right. \]

\begin{theorem}
This is the first theorem.
\end{theorem}

\begin{proof}
This is the proof of the first theorem. We show how to write pseudo-code now.
%*** USE PSEUDO-CODE ONLY IF IT IS CLEARER THAN AN ENGLISH DESCRIPTION

Consider a comparison between $x$ and~$y$:
\begin{tabbing}
\hspace*{.25in} \= \hspace*{.25in} \= \hspace*{.25in} \= \hspace*{.25in} \= \hspace*{.25in} \=\kill
\>{\bf if} $x$ or $y$ or both are in $S$ {\bf then } \\
\>\> answer accordingly \\
\>{\bf else} \\
\>\>    Make the element with the larger score (say $x$) win the comparison \\
\>\> {\bf if} $F(x) + F(y) < \frac{n}{t-1}$ {\bf then} \\%
\>\>\> $F(x) \leftarrow F(x) + F(y)$ \\
\>\>\> $F(y) \leftarrow 0$ \\
\>\> {\bf else}  \\
\>\>\> $S \leftarrow S \cup \{ x \} $ \\
\>\>\> $r \leftarrow r+1$ \\
\>\> {\bf endif} \\
\>{\bf endif} 
\end{tabbing}

This concludes the proof.
\end{proof}

\end{document}
