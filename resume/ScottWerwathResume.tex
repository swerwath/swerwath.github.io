%%%%%%%%%%%%%%%%%%%%%%%%%%%%%%%%%%%%%%%%%
% Medium Length Professional CV
% LaTeX Template
% Version 2.0 (8/5/13)
%
% This template has been downloaded from:
% http://www.LaTeXTemplates.com
%
% Original author:
% Trey Hunner (http://www.treyhunner.com/)
%
% Important note:
% This template requires the resume.cls file to be in the same directory as the
% .tex file. The resume.cls file provides the resume style used for structuring the
% document.
%
%%%%%%%%%%%%%%%%%%%%%%%%%%%%%%%%%%%%%%%%%

%----------------------------------------------------------------------------------------
%	PACKAGES AND OTHER DOCUMENT CONFIGURATIONS
%----------------------------------------------------------------------------------------
\documentclass{resume} % Use the custom resume.cls style

\usepackage[left=0.5in,top=0.6in,right=0.65in,bottom=0.65in]{geometry} % Document margins
\usepackage{ textcomp }
\usepackage{xfrac}

\name{Scott Werwath} % Your name
\address{(804) 380-1188 \\ sw@swerwath.com \\ swerwath.com} % Your phone number an} % Your address

\begin{document}

%----------------------------------------------------------------------------------------
%	TECHNICAL STRENGTHS SECTION
%----------------------------------------------------------------------------------------

%\begin{rSection}{Technical Skills}
%\begin{tabular}{ @{} >{\bfseries}l @{\hspace{6ex}} l }
%Languages & C++, Python, C, Objective C, Javascript \\
%Frameworks & Keras, Tensorflow, MapReduce, NumPy, Node, .NET \\
%Misc. & Git, Deep Learning, NLP, WebSockets, Relational Databases
%\end{tabular}

%\end{rSection}

%----------------------------------------------------------------------------------------
%	EDUCATION SECTION
%----------------------------------------------------------------------------------------

\begin{rSection}{Education}
{\bf University of California, Berkeley} \hfill {September 2015\textminus  December 2018} \\ 
B.S. Electrical Engineering \& Computer Sciences \hfill {\em GPA (major): 3.80, GPA (overall): 3.65}
\end{rSection}

%----------------------------------------------------------------------------------------
%	WORK EXPERIENCE SECTION
%----------------------------------------------------------------------------------------

\begin{rSection}{Experience}
\begin{rSubsection}{Fathom Health}{March 2020 \textminus Present}{Software Engineer}{San Francisco, California}
\item[] Owned and led monitoring, alerting and system resiliency projects and on-call processes to ensure compliance with customer SLAs as production ML systems scaled over 1000x
\item[] Created production deep learning NLP models trained on mutli-terabyte datasets of medical documents, including dataset creation, cleaning, synthetic example generation, model tuning, training, and evaluation, and automated checks on model predictions in production to ensure accuracy
\item[] Designed engineering processes to rapidly integrate with new enterprise customers as company scaled
\item[] Technology Used: Python, Spark, Airflow, GCP, Tensorflow, Kubernetes
\end{rSubsection}
\begin{rSubsection}{The Human Diagnosis Project}{January 2019 \textminus Present}{Software Engineer}{San Francisco, California}
\item[] Created online models to measure the clinical reasoning abilities of physicians as they solved patient cases
\item[] Overhauled recommendations engine to serve teaching cases to physicians who would find them challenging and engaging, and to triage patient cases to physicians who would be best suited to solve them
\item[] Rewrote core parts of mobile app and backend API to allow use in offline/low internet conditions
\item[] Technology Used: Python, Django, Docker, GCP, Tensorflow, React Native
\end{rSubsection}
\begin{rSubsection}{Facebook}{May 2018 \textminus August 2018}{Software Engineering Intern}{New York, New York}
\item[] Built on-client caching system for iOS clients of a cross-platform UI framework to reduce amount of source code sent over the network
\item[] Added type system to framework to extract types in Flow JavaScript and statically enforce cross-language type safety between serverside Hack code and the JavaScript it interacts with
\item[] Technology Used: Hack, Objective C, C++, GraphQL, OCaml, JavaScript
\end{rSubsection}
\begin{rSubsection}{Facebook}{May 2017 \textminus August 2017}{Software Engineering Intern}{Seattle, Washington}
\item[] Designed and built service to parse binaries, cache their symbol tables, and efficiently serve requests for symbolization of address stacks
\item[] Integrated new service into profiling tool deployed across every host in Facebook's fleet, reducing its p90 memory usage by 20\% and allowing for the use of more accurate sampling techniques
\item[] Technology Used: C++, Thrift
\end{rSubsection}
\begin{rSubsection}{Google}{January 2017\textminus May 2017}{Software Engineering Intern}{Mountain View, California}
\item[] Developed NLP techniques to disambiguate entity mentions in unstructured text based on linguistic context
\item[] Wrote large-scale data processing pipelines for example generation, model training, and model evaluation
\item[] Technology Used: C++, Python, NumPy, MapReduce, TensorFlow
\end{rSubsection}
\begin{rSubsection}{SolarCity}{June 2016\textminus August 2016}{Software Engineering Intern}{San Francisco, California}
\item[] Designed and built Node.js WebSocket microservice to enable real time interaction and data streaming between customers and sales representatives
\item[] Refactored .NET routes and database schemas, reducing average customer-facing API response time by 75\%
\item[] Technology Used: C\#, .NET, Node.js, Websockets, SQL, Redis
\end{rSubsection}
\end{rSection}

\begin{rSection}{Research}
\begin{rSubsection}{UC San Francisco, Department of Radiology}{September 2017\textminus January 2019}{}{}
\item[] Designed and implemented NLP model to automatically categorize free-text radiology reports based on whether or not they contain urgent findings
\item[] Technology Used: Python, Keras, Tensorflow, NLTK
\end{rSubsection}
\end{rSection}


%----------------------------------------------------------------------------------------
%	EXAMPLE SECTION
%----------------------------------------------------------------------------------------

%\begin{rSection}{Section Name}

%Section content\ldots

%\end{rSection}

%----------------------------------------------------------------------------------------

\end{document}
